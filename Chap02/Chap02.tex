\documentclass[a4paper, 12pt]{article}
\usepackage[english]{babel}
\usepackage[utf8]{inputenc}
\usepackage[margin=1in]{geometry}
\usepackage{tcolorbox}
\usepackage{listings}
\usepackage{amsmath, amssymb}

\begin{document}
\title{\textbf{Chapter 2: Exercises}}
\author{Raymart Jay E. Canoy}
\date{\today}
\maketitle

\noindent 1. Give R assignment statements that set the variable $z$ to
\begin{itemize}
\item[(a)]{${x^{a}}^{b}$}
\item[(b)]{$(x^{a})^{b}$}
\item[(c)]{$3x^{3}+2x^{2}+6x+1$}
\item[(d)]{the digit in the second decimal place of $x$}
\item[(e)]{$z+1$}
\end{itemize}

\noindent 2. Give R expression that return the following matrices and vectors
\begin{itemize}
\item[(a)]{$(1,2,3,4,5,6,7,8,7,6,5,4,3,2,1)$}
\item[(b)]{$(1,2,2,3,3,3,4,4,4,4,5,5,5,5,5)$}
\item[(c)]{$\begin{pmatrix}
0 & 1 & 1\\
1 & 0 & 1\\
1 & 1 & 0
\end{pmatrix}$}
\item[(d)]{$\begin{pmatrix}
0 & 2 & 3 \\
0 & 5 & 0 \\
7 & 0 & 0
\end{pmatrix}$}
\end{itemize}


\noindent 3. Suppose $\mathtt{vec}$ is strictly positive vector of length 2. Interpreting $\mathtt{vec}$ as coordinates of a point $R^{2}$, use R to express it in polar coordinates.


\noindent 4. Use R to produce a vector containing all integers from 1 to 100 that are not divisible by 2, 3, or 7.

\noindent 5. Suppose that queue <- c("Steve", "Russel", "Alison", "Liam") and that queue represents a supermarket queue with Steve first in line. Using R expressions update the supermarket queue as successively:
\begin{itemize}
\item[(a)]{Barry arrives;}
\item[(b)]{Steve is served;}
\item[(c)]{Pam talks her way to the front with one item;}
\item[(d)]{Barry gets impatient and leaves;}
\item[(e)]{Alison gets impatient and leaves}
\end{itemize}
\noindent For the last case, you should not assume that you know where in the queue Alison is standing. Finally, using the function $\mathtt{which(x)}$, find the position of Russell in the queue. Note that when assigning a text string to a variable, it needs to be in quotes.

\noindent 6. Which of the following assignments will be successful? What will the vectors $x$, $y$, and $z$ look at each stage?
\begin{itemize}
\item{$\mathtt{rm(list) = ls()}$}
\item{$x \leftarrow 1$}
\item{$x[3] \leftarrow 3$}
\item{$y \leftarrow c()$}
\item{$y[2] \leftarrow 2$}
\item{$y[3] \leftarrow y[1]$}
\item{$y[2] \leftarrow y[4]$}
\item{$z[1] \leftarrow 0$}
\end{itemize}

\noindent 7. Build a $10 \times 10$ identity matrix. Then make all the non-zero elements 5. Do this latter step in at least two different ways.
\end{document}